\documentclass[11pt,a4paper,sans]{moderncv}

%% ModernCV themes
\moderncvstyle{classic}
\moderncvcolor{blue}
\renewcommand{\familydefault}{\sfdefault}
\nopagenumbers{}

%% Character encoding
\usepackage[utf8]{inputenc}
%% Adjust the page margins
\usepackage[scale=0.84]{geometry}
%% For company logos
\usepackage{graphicx}
\graphicspath{{./images/}}

%% Personal data
\firstname{Jakub}
\familyname{Sokołowski}
\title{Administrator Systemów i Developer}
\address{Klaudyny 32 lok 1}{Warszawa, 01-684}{Polska}
\mobile{+48 600 858 250}
\email{jakub@gsokolowski.pl}
\homepage{gsokolowski.pl}
\extrainfo{Urodzony 01.02.1988}
\photo[74pt][0.5pt]{me}

\begin{document}
\makecvtitle

\section{Edukacja}
    \cventry{2007--2013}{Inżynier}{Polsko-Japońska Wyższa Szkoła Technik Komputerowych}{Warszawa}{\textit{Informatyka}}{Studia inżynierskie ukończone w katerze Sieci Komputerowych na specjalizacji ``Programowanie Systemowe i Sieciowe''.}

    \vspace*{3mm}
    \cventry{2005--2007}{Wykształcenie średnie}{Liceum Ogólnokształcące im. Joachima Lelewela}{Warszawa}{}{}

\section{Doświadczenie}
\cventry{2015--Now\\\includegraphics[width=\hintscolumnwidth]{aws}}{System Development Engineer}{Amazon Web Services}{Dublin}{Pełen etat}
{W zespole odpowiedialnym za reagowanie na wydażenia w globalnej infrastrukturze AWS oraz tworzenie narzędzi do zarządzania takimi wydażeniami. Odpowiedzialny za:}
\cvlistitem{Zebranie i prowadzenie zespołu w celu rozwiązania problemów wszelakiej natury}
    \cvlistitem{Wyznaczanie zakresu projektów oraz architektura systemów}
    \cvlistitem{Implementacja systemów do wykrywania i przewidywania wydażeń}
    \cvlistitem{Automatyzacja tworzenia infrastruktur dla nowych systemów}
    \cventry{2014--2015\\\includegraphics[width=\hintscolumnwidth]{codility}}{System Administrator}{Codility Polska Sp. z o.o}{Warsaw}{Pełen etat}{Zarządzanie wewnętrznymi narzędziami deweloperskimi oraz administracja i rozwój infrastruktury głównej usługi Codility. Odpowiedzialny za:}
\cvlistitem{Zarządzanie serwerami przy pomocy Chef i automatyzacja wdrażenia przy pomocy Fabric.}
    \cvlistitem{Utrzymanie systemów monitorowania(Icinga, Grafana, Sentry, Pingdom)}
    \cvlistitem{Utrzymanie systemów logowania(logstash, elasticsearch, Kibana)}
    \cvlistitem{Integracja wewnętrznych systemów developerskich(Trac, GitLab, Jenkins)}
    \cvlistitem{Konsultacje z klientami na temat bezpieczeństwa danych oraz prywatności}
\cventry{2013--2014\\\includegraphics[width=\hintscolumnwidth]{ascen}}{Administrator Systemów i Deweloper}{Ascen Sp. z o.o}{Warszawa}{Pełen etat}{Administracja oraz tworzenie rozwiązań transformacji danych oraz raportowania na potrzeby wielu projektów. Między innymi:}
    \cvlistitem{Rozwiązania ETL dla Polskich Linii Lotniczych LOT. W tym ładowanie i eksportowanie danych w wielu formatach w Microsoft SSIS oraz implementacja systemu logowania.}
    \cvlistitem{Systemu raportowego dla Urzędu do Spraw Cudzoziemców. Tworzenie pełnego planu kopii bezpieczeństwa oraz testowanie i rozwiązywanie problemów wydajności z SSRS i SharePoint}
    \cvlistitem{Infrastruktury do współdzielenia farmy renderującej 3D dla Nemotion Studios. Automatyzacja uruchamiania zadań Autodesk Backburner. Konfiguracja dostępu poprzez FTP oraz WebDAV}
    \cvlistitem{Wsparcie produktu IBM DataStage dla TUiR Warta. Rozwiązywanie problemów komunikacji baz danych. Diagnozowanie problemów IBM Information Server}
\cventry{2013--2014\\\includegraphics[width=\hintscolumnwidth]{raiffeisen}}{Administrator Systemów}{Raiffeisen Polbank S.A.}{Warszawa}{Pełen etat}{Administrowanie środowiskami migracyjnymi opartymi na systemach AIX Unix udostępniającymi usługi IBM DB2 oraz narzędzia migracyjne IBM oraz Oracle na potrzeby przeniesienia danych z systemów bazodanowych Polbank do RaiffeisenPolbank. Wykonywanie głównego procesu migracji oraz komunikacja z koordynatorami oraz grupami projektowymi. Odpowiedzialny za:}
    \cvlistitem{Administracja Serwerami IBM AIX Unix, Oracle Data Integrator i IBM DataStage.}
    \cvlistitem{Analizowanie i diagnostyka problemów w bazach danych SQL}
    \cvlistitem{Zarządzanie wymianą plików pomiędzy środowiskami migracyjnymi i grupami projektowymi}
    \cvlistitem{Automatyzacja zadań przy pomocy skryptów(Bash, AWK, SQL)}
\cventry{2011--2013\\\includegraphics[width=\hintscolumnwidth]{trinity}}{Specjalista IT}{Trinity Corporate Services Sp. z o.o.}{Warszawa}{Pełen etat}{Nadzorowanie i opieka nad infrastrukturą IT oraz rozwiązywanie wszelkich problemów technicznych. Wykonywanie projektów, np.: migracji systemu pocztowego z lokalnych serwerów do zdalnego dostawcy, modernizacji infrastruktury biur zagranicznych lub wdrażanie oprogramowania. Odpowiedzialny za:}
    \cvlistitem{Serwery Windows(2003 oraz 2008) oraz Linux(Ubuntu Server)}
    \cvlistitem{Monitorowanie działalności systemów informatycznych(Cacti, Graylog2)}
    \cvlistitem{Automatyzację wszelkich procesów informatycznych(kopie bezpieczeństwa, konwersje danych)}
    \cvlistitem{Infrastrukturę fizyczną środowiska IT(serwery, routery, komputery biurowe)}
    \cvlistitem{Zarządzenie usługami takimi jak: MS SQL, MS Exchange, MySQL, Hyper-V, Postfix, Git}

\section{Praca Inżynierska}
    \cvitem{Tytuł}{\textbf{Udostępnianie Znakowych Urządzeń za Pośrednictwem Sieci w Systemach GNU/Linux}}
    \cvitem{Specjalizacja}{Programowanie Systemowe i Sieciowe\hspace{20mm}Ocena \textbf{5.0}}
    \cvitem{Opis}{Zestaw oprogramowania składający się z serwera oraz modułu jądra Linux udostępniający w sposób przezroczysty dla użytkowników urządzenia znakowe na zdalnych maszynach.}

\section{Języki obce}
    \cvitemwithcomment{Angielski}{Biegle}{Łatwość w bezpośredniej komunikacji i szerokie słownictwo techniczne.}
    \cvitemwithcomment{Niemiecki}{Podstawy}{Podstawowa znajomość nabyta w liceum.}

\section{Umiejętności}
\cvitem{Języki}{C, C\#, Python, Ruby, JavaScript, Java, Bash, AWK, Powershell, SQL, Lua, \LaTeX}
\cvitem{Systemy}{GNU/Linux(Debian, Ubuntu i Getoo, Amazon Linux), Windows Server, IBM AIX Unix}
    \cvitem{Sieć}{Routing statyczny oraz dynamiczny RIP i OSPF, DHCP, DNS, NAT, QoS, VPN, VLAN}
    \cvitem{Narzędzia}{Git, GNU Make, Vim, Microsoft RSAT, MS Powershell}
    \cvitem{Platformy}{Chef, Ansible, Fabric, Apache, nginx, lighttpd, Gunicorn, uWSGI, Samba, NFS, Cacti, Graylog2, iptables, bind, postfix, Active Directory, Group Policy, Hyper-V, VMWare, VirtualBox, DB2, MSSQL, MySQL, Postgresql, IIS}
    \cvitem{AWS}{EC2, Route53, CloudFront, S3, Glacier, RDS, IAM, SES, SQS, Lambda}
    \cvitem{WebDev}{React, Redux, Backbone.js, GraphQL, Relay}
    \cvitem{Extra}{IBM DataStage, SQL Server Integration Services and Reporting Services, SharePoint}

\section{Kursy i certyfikaty}
    \cvitem{2014}{Ukończony kurs Oracle Exadata Technical Boot Camp}
    \cvitem{2014}{Ukończony kurs IBM InfoSphere Guardium Data Security}

\lfoot{\scriptsize{Wyrażam zgodę na przetwarzanie moich danych osobowych zawartych w mojej aplikacji dla potrzeb niezbędnych do realizacji procesów rekrutacji (zgodnie z Ustawą z dnia 29 sierpnia 1997 r. o ochronie danych osobowych tj. Dz. U. z 2002 r., Nr 101, poz. 926, ze zm.).}}
\end{document}
