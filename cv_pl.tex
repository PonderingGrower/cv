\documentclass[11pt,a4paper,sans]{moderncv}

%% ModernCV themes
\moderncvstyle{classic}
\moderncvcolor{blue}
\renewcommand{\familydefault}{\sfdefault}
\nopagenumbers{}

%% Character encoding
\usepackage[utf8]{inputenc}

%% Adjust the page margins
\usepackage[scale=0.75]{geometry}

%% Personal data
\firstname{Jakub}
\familyname{Sokołowski}
\title{Specjalista IT}
\address{Warszawa}{Klaudyny 32 m 1}
\mobile{600~858~250}
\phone{+48~(22)~833~77~50}
\email{panswiata@gmail.com}
\homepage{github.com/PonderingGrower}
\extrainfo{Urodzony 01.02.1988}
\photo[64pt][0.4pt]{me}

\begin{document}
\makecvtitle

\section{Edukacja}
    \cventry{2007--2013}{Inżynier}{Polsko-Japońska Wyższa Szkoła Technik Komputerowych}{Warszawa}{\textit{Informatyka}}{Studia inżynierskie ukończone w katerze Sieci Komputerowych na specjalizacji ``Programowanie Systemowe i Sieciowe''.}

    \vspace*{4mm}
    \cventry{2005--2007}{Wykształcenie średnie}{Liceum Ogólnokształcące im. Joachima Lelewela}{Warszawa}{}{}

\section{Praca Inżynierska}
\cvitem{Tytuł}{\textbf{Udostępnianie znakowych urządzeń fizycznych za pośrednictwem sieci w systemach GNU/Linux}}
    \cvitem{Specjalizacja}{Programowanie Systemowe i Sieciowe}
    \cvitem{Opis}{Zestaw oprogramowania składający się z serwera oraz modułu jądra Linux udostępniający w sposób przezroczysty dla użytkowników urządzenia znakowe na zdalnych maszynach.}

\section{Języki obce}
    \cvitemwithcomment{Angielski}{Biegle}{Łatwość w bezpośredniej komunikacji w języku angielskim. Dobra znajomość słownictwa technicznego.}

\section{Umiejętności}
    \cvitem{Języki}{C, C++, Java, bash, Powershell, SQL, Lua, \LaTeX, HTML, CSS, UML}
    \cvitem{Systemy}{GNU/Linux(wszystkie odmiany Debian oraz Gentoo), Windows, Windows Server 2003 i 2008}
    \cvitem{Sprzęt}{Komputery i peryferia, Serwery oraz stacje robocze DELL, Przełączniki oraz routery CISCO i DELL, Zasilacze awaryjne APC}
    \cvitem{Platformy}{Active Directory, Group Policy, Hyper-V, VirtualBox, MSSQL, Mysql, Postgresql, IIS, apache, nginx, lighttpd, Samba, NFS, Cacti, Graylog2}
    \cvitem{Narzędzia}{Microsoft RSAT, MS Powershell, MS SQL Studio, Pakiet MS Office, Git, GNU Make, VIM, Gimp, Photoshop}
    \cvitem{Specjalistyczne}{Oprogramowanie ERP Exact Globe, Systemy księgowania Sage, System księgowo-płacowy Płatnik, Większość europejskich platform bankowości elektronicznej}

\section{Doświadczenie}
\cventry{2011--2013}{Specjalista IT}{Trinity Corporate Services Sp. z o.o.}{Warszawa}{Pełen etat}{Nadzorowanie oraz opieka nad infrastrukturą IT oraz rozwiązywanie wszelkich problemów z nią związanych. Wykonywanie projektów informatycznych, np.: migracji systemu pocztowego z lokalnych serwerów do zdalnego dostawcy, modernizacji infrastruktury biur zagranicznych lub wdrażanie nowego oprogramowania w biurach.}
    \cvitem{}{Odpowiedzialny za:}
    \cvlistitem{Serwery Windows(2003 oraz 2008) oraz Linux(Ubuntu Server)}
    \cvlistitem{Monitorowanie działalności systemów informatycznych(Cacti, Graylog2)}
    \cvlistitem{Automatyzację wszelkich procesów informatycznych(tworzenia kopii bezpieczeństwa, konwersji danych)}
    \cvlistitem{Infrastrukturę fizyczną środowiska IT(serwery, routery, komputery biurowe)}
    \cvlistitem{Zarządzenie bazami danych MSSQL oraz Mysql}
    \cvlistitem{Zarządzenie serwerem poczty Postfix}
    \cvlistitem{Obsługiwanie serwerów poczty Microsoft Exchange oraz Hosted Exchange}
    \cvlistitem{Odbieranie, rejestrowanie i rozwiązywanie problemów z infrastrukturą IT}
    \cvlistitem{Wsparcie techniczne użytkowników oraz klientów}

    \vspace*{4mm}
    \cventry{2012}{Umowa o dzieło}{Empirius Group Sp. z o.o.}{Warszawa}{Praca doraźna}{Stworzenie od podstaw infrastruktury komputerowej w małym biurze firmy dla około 10 komputerów. Wybór sprzętu i oprogramowania, instalacja i konfiguracja, rozłożenie kabli. W tym instalacja oraz konfiguracja serwera HP z systemem Windows Server 2008 na potrzeby bezpiecznej wymiany plików oraz współdzielenia drukarki oraz skanera.}

    \vspace*{4mm}
    \cventry{2010}{Stażysta-Praktykant}{BIS Izomar Sp. z o.o.}{Warszawa}{Dwumiesięczny staż}{Staż skupiający się na obsłudze technicznej placówki budowlanej w elektrociepłowni ``Żerań''. Odpowiedzialny za komputery oraz drukarki. W drugiej połowie stażu powierzone wykonanie konfiguracji VLAN na przełącznikach trzeciej warstwy firmy DELL na potrzeby rozdzielania sieci głównego biura firmy w celu zwiększenia bezpieczeństwa oraz ułatwienia zarządzania.}
    \cvitem{}{Zdobyte doświadczenie:}
    \cvlistitem{Struktura firmy i jej działanie - Roboty budowano-montażowe}
    \cvlistitem{Przepisy o ochronie danych osobowych oraz tajemnic państwowych oraz służbowych}
    \cvlistitem{Tworzenia infrastruktury IT małego biura na budowie ``Żerań''}

\section{Kursy i certyfikaty}
    \cvitem{2011}{PJWSTK Certificate in Business English, poziom C1}
    \cvitem{2009}{Ukończony kurs CISCO CCNA}
    \cvitem{2009}{Ukończony kurs pierwszego stopnia Linux Professional Institute}

\section{Zainteresowania}
    \cvitem{}{Informatyka, ekonomia, historia, socjologia, filozofia, futurologia, astronomia, biologia molekularna}

\lfoot{Wyrażam zgodę na przetwarzanie moich danych osobowych zawartych w ofercie przy realizacji procesów rekrutacji i selekcji zgodnie z przepisami ustawy z 29.08.1997 r. o ochronie danych osobowych (Dz. U. z 2002r. Nr 101, poz. 926).}
\end{document}
